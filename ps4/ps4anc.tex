\documentclass[a4paper,11pt]{amsart}
\usepackage{amssymb}
\usepackage{graphicx}

\parskip 1ex
\parindent 0 pt

\newcounter{temp}
\newcounter{prob_counter}
\newcounter{sprob_counter}

\newenvironment{problem}
{\begin{list}{{\bf \arabic{prob_counter}}}{
      \usecounter{prob_counter}
      \addtolength{\labelsep}{.6ex}
      \addtolength{\itemsep}{4.3ex}
      \setlength{\leftmargin}{1.4em}}
      \setcounter{prob_counter}{\value{temp}}
}
{\setcounter{temp}{\value{prob_counter}}
  \end{list}
}

\newenvironment{subprob}
{
  \begin{list}{{\bf \alph{sprob_counter}}}{
      \usecounter{sprob_counter}
      \addtolength{\labelsep}{.6ex}
      \addtolength{\itemsep}{.5ex}
      \setlength{\leftmargin}{1.7em}}
}
{\end{list}}

\newenvironment{solution}{\textbf{Solution.}}{\qed}

\newcommand{\rubrik}[1]{\bigskip \begin{center}{\bf #1}\end{center} \medskip}

\newcommand{\NN}{\mathbb{N}}
\newcommand{\ZZ}{\mathbb{Z}}
\newcommand{\QQ}{\mathbb{Q}}
\newcommand{\RR}{\mathbb{R}}




\begin{document}

\pagestyle{empty}
\thispagestyle{empty}

{\small{\sc\noindent
        Arnaldur Bjarnason ({\tt arnaldur15@ru.is}) and Árni Dagur Guðmundsson ({\tt arnidg@protonmail.ch})
}}

\rubrik{PROBLEM SET 4 (T-445-GRTH)}

You need to collect $\bf 65$ points to get a full score {\bf but} you cannot get more than {\bf X} points (in total) from a problem section with annotation {\bf max X}.



\section{Planar Graphs ({\bf max 30})}

\begin{problem}
\item (5 points) Let G be a simple planar bipartite graph with bipartition $V(G) = X \cup Y$.
Show that  $|E(G)| \le 2 |X| + 2|Y| - 4$.
\end{problem}
\begin{solution}
Bipartite graphs can not have a 3-cycle as they can not have a $n$-cycle where $n$ is odd. Corollary 7.15 states that a simple planar graph $G$ has the property $|E(G)| \leq 2|V(G)| - 4$ and as $2|X| + 2|Y| = 2|V(G)|$ the equation is true.
\end{solution}

\begin{problem}
\item (7 points) Let $G$ be a graph such that for every two vertices $u, v$, there
are at most two vertex-disjoint paths of positive lengths from $u$ to $v$.
Show that $G$ is planar.
\end{problem}


\begin{problem}
\item (5 points) A plane graph is self-dual if it is isomorphic to its dual graph.
Show that, if $G$ is self-dual, then $2 |V(G)|$ = $|E(G)| + 2$.
\end{problem}

\begin{problem}
\item (12 points) Let $G$ and $G^*$ be a plane graph and its dual. Show that:
\begin{subprob}
\item If $\chi(G^*)=2$, then $d_G(u)$ is even for all $u\in V(G)$,
\item An Eulerian graph cannot have an odd-size edge cut (cf. Thm. 6.8),
\item If $G$ is Eulerian, $G^*$ has no odd cycles,
\item Derive the conclusion: plane graph $G$ is Eulerian iff $G^*$ is bipartite.
\end{subprob}
\end{problem}

\begin{problem}
\item (8 points) Show that for two homeomorphic graphs $G$ and $G'$,
\[
|V(G)| - |E(G)| = |V(G')| - |E(G')|.
\]
\end{problem}
\begin{solution}
  For $G$ to be homeomorphic to $G'$, there must be a graph $M$ that is a minor of $G$ and a graph $M'$ that is a minor of $G'$ such that $M$ is isomorphic to $M'$. $M$ and $M'$ must have been constructed from $G$ and $G'$ only using smoothing, which is the process of contracting an edge where one vertex has degree 2. Smoothing can also be considered as removing a vertex of degree 2 and connecting its neighbours. $|V(G)| - |E(G)| = |V(M)| - |E(M)|$ this must be true as a contraction reduces the number of vertices by one and the number of edges as well and $M$ was created using contractions only. As $M$ is isomorphic to $M'$, $|V(M)| - |E(M)| = |V(M')| - |E(M')|$ is a fact and thus $$|V(G)| - |E(G)| = |V(G')| - |E(G')|$$
\end{solution}
\begin{problem}
\item (8 points) Consider a finite collection of straight lines in the plane and the geographical map formed by these lines. Show that the regions of this map can be properly colored using two colors.
\end{problem}

\begin{solution}
  Let's prove this by induction:

  \textbf{Base case:}
  We start with only one face -- the infinite face. Draw any line, and by
  inverting the colour of one side of the line, the property holds; we can
  colour the plane with only two colours.

  \textbf{Inductive step:}
  Assume we have a plane whose faces can be 2-coloured. Draw any line $l$ on the
  plane. If we do so, we can trivially observe that every pair of faces
  seperated by $l$ share the same colour; this means that if we invert the
  colours on one side of $l$, we have a proper 2-colouring of the plane.

\end{solution}

\section{Coloring 1 ({\bf max 15})}


\begin{problem}
\item (9 points)
\begin{subprob}
\item Determine (and justify) the chromatic numbers of the Petersen and Gr\"otzsch graphs (in the same order below)
\begin{center}
\includegraphics[height=4.5cm]{petersen.pdf}
\includegraphics[height=4.5cm]{grotzsch.pdf}
\end{center}
\item Show that the \emph{rigid ladder graph} has chromatic number 3
\begin{center}
\includegraphics[height=1.5cm]{ladder.pdf}
 \end{center}
\end{subprob}
\end{problem}

\begin{solution}
Since verteces 1-5 form an odd cycle, the chromatic number of the Petersen graph $P$
is $3 \leq \chi(P)$. Furthermore, by Brook's theorem, we have that $\chi(P) \leq
\Delta(P) = 3$, so $\chi(P) = 3$.
\end{solution}

\begin{solution}
Since verteces 1-5 form an odd cycle, the chromatic number of the Grötzsch graph
$G$ is $3 \leq \chi(G)$. We can easily find a colouring of the $G$ with four
colours (see picture), so $\chi(G) \in \{3, 4\}$.

Let's assume that there exists a valid 3-colouring of $G$. Since you need 3
colours to colour verteces 1-5, and there is a 1-1 correspondence between the
colours of 1-5 and 6-10, you need 3 colours to colour verteces 6-10. This is a
contradiction since vertex 11 is connected to each of verteces 6-10, and thus
needs a new colour. The chromatic number is 4.
\end{solution}

\begin{solution}
  TODO (section b)
\end{solution}

\begin{problem}
\item (8 points) Show that:
\begin{subprob}
\item If $G$ is a simple graph on $n$ vertices that is not $K_n$,
then $\chi(G) \le n-1$.
\item If $G$ is a simple graph on $n$ vertices with $m \le \frac{(n-1)(n-2)}{2} - 1$ edges, then $\chi(G) \le n-2$.
\end{subprob}
\end{problem}


\begin{problem}
\item (8 points) Consider a graph $G$ with $\chi(G)=k$. Show that for every $1\le k'\le k-1$, there are graphs $G_1,G_2$, such that $\chi(G_1)=k'$, $\chi(G_2)=k-k'$, and $G=G_1\cup G_2$.
{\small Note: This problem is ill-stated. But we can save the day by assuming
  that $1 < k' < k-1$, i.e., we have strict inequalities.}
\end{problem}


\section{Coloring 2 ({\bf max 60})}


\begin{problem}
\item (8 points) Let $G$ and $H$ be simple graphs. Define the \emph{Cartesian product} $G\times H$ as the simple graph with vertex set $V(G)\times V(H)$ and edge set
\begin{align*}
E(G\times H)=\{\{u,v\}, \{u',v'\} : u&=u'\text{ and }\{v, v'\}\in E(H)\text{ or }\\
 v&=v'\text{ and }\{u, u'\}\in E(G)\}.
\end{align*}
Show that $\chi(G\times H)=\max\{\chi(G), \chi(H)\}$.
\end{problem}



\begin{problem}
\item (7 points) Show that for a simple graph $G$ with $m$ edges,
\\$\chi(G)\le \frac{1}{2} + \sqrt{2m + \frac{1}{4}}.$\\
{\small Hint: First show that $m \ge {\chi(G) \choose 2}$, by considering an optimal coloring.}
\end{problem}

\begin{solution}
For every colour $1$ and $2$, there must be a exist a pair of verteces $v, u \in
V(G)$ such that $c(v) = 1$ and $c(u) = 2$, since otherwise the two colours could
be combined. From that we have that
\[
  m \geq {\chi(G) \choose 2} = \frac{1}{2}\chi(G)(\chi(G) - 1)
  = \frac{1}{2}\chi(G)^{2} - \frac{1}{2}\chi(G)
\]
and thus:
\[
  2m \geq \chi(G)^{2} + \chi(G)
\]
If we complete the square, we have:
\[
  2m \geq (\chi(G) - \frac{1}{2})^{2} + \frac{1}{4}
\]
The rest is simple algebra:
\[
  \sqrt{2m + \frac{1}{4}} + \frac{1}{2} \leq \chi(G)
\]
\end{solution}

\begin{problem}
\item (9 points) Let $G_1$ and $G_2$ be subgraphs of a given simple graph. Show that $\chi(G_1 \cup G_2) \le \chi(G_1) \cdot \chi(G_2)$.
\end{problem}

\begin{problem}
\item (8 points) Let $G$ be a graph and let $\ell$ be the length of a longest path in $G$.
Show that $\chi(G) \le \ell + 1$.\\
{\small Hint: Think about the end-vertices of longest paths (and their degrees).}
{\small Another hint: $D(G) \leq \ell$}
\end{problem}

\begin{problem}
\item (8 points) Find a lower bound on inductiveness in terms of average degree.
{\small Hint: You can use a problem from a previous problem set.}
\end{problem}

\begin{problem}
\item (8 points) Find a graph where inductiveness gives a much better bound than Brooks' theorem. How large can $\frac{\Delta(G)}{D(G)}$ get (as a function of $\Delta$)?
\end{problem}


\begin{problem}
\item (20 points) Let $G$ be a simple graph on $n$ vertices and $\overline{G}$ be its complement. Show that $2\sqrt{n} \le \chi(G) + \chi(\overline{G}) \le n+1$.\\
{\small Hint: For the first inequality, use the AM-GM inequality. For the second one, you can try the following induction. Let $u\in V(G)$. If $G' = G - u$, then $\chi(G)\le \chi(G')+1$ and $\chi(\overline{G})\le \chi(\overline{G'})+1$. Show that if equality holds in both cases, then $d_{G}(u)\ge \chi(G)$ and $d_{\overline{G}}(u)\ge \chi(\overline G)$ (what next?). Otherwise, proceed by induction.}
\end{problem}

\begin{solution}
  Let's first show that $2\sqrt{n} \leq \chi(G) + \chi(\overline{G})$:

  By the AM-GM inequality:
  \[
  2\sqrt{\chi(G)\chi(\overline{G})} \leq \chi(G) +
  \chi(\overline{G})
  \]
  Furthermore, $n \leq \chi(G)\chi(\overline{G})$. Why? Let $\chi(G) = k$, that
  means that there is at least one independent set $S$ of verteces (verteces that
  share the same colour), with at least $\frac{n}{k}$ verteces.
  Since the induced subgraph $G[S]$ is a null graph, $\overline{G}[S]$ is a
  complete graph, so $\chi(\overline{G}) \leq \frac{n}{k}$; the chromatic number
  of a graph must be greater than or equal to its clique number. So
  $n = k*\frac{n}{k} \leq \chi(G)\chi(\overline{G})$. So far we have:
  \[
    2\sqrt{n} \leq 2\sqrt{\chi(G)\chi(\overline{G})} \leq \chi(G) + \chi(\overline{G})
  \]

  \textbf{Base case:}
  TODO

  \textbf{Induction step:}
  Let G be a graph with $n$ vertices. Let $v \in V(G)$. By our hypothesis, if
  $\chi(G - v) = k$ and $\chi(\overline{G} - v) = l$, then $k + l = n$.

  Let's observe these two cases:
  \begin{enumerate}
  \item If $d_G(v) < k$ then you can colour $G$ with $k$ colours. In $\overline{G}$ we
    assign $v$ a new colour, since $G$ may be harder to colour than $G - v$. Ergo,
    \[
      \chi(G) + \chi(\overline{G}) \leq k + (l + 1) = n + 1
    \]

  \item We can follow the same reasoning if $d_G(v) \geq k$: In that case,
    $d_{\overline{G}}(v) \leq (n - 1) - k = l - 1 < l$. We now have the same as
    above; $\overline{G}$ can be coloured with $l$ colours, and if we assign $v$
    a new colour in $G$, we have:
    \[
      \chi(G) + \chi(\overline{G}) \leq (k + 1) + l = n + 1
    \]
  \end{enumerate}

\end{solution}

\begin{problem}
\item (10 points) Let G be a graph such that every two odd cycles have at least one common vertex.
Prove that G is 5-colorable.
\end{problem}



\begin{problem}
\item (8 points) Show that if $G$ is regular with degree 3 and  Hamiltonian then $\chi'(G)\le 3$.
\end{problem}




\end{document}


%%% Local Variables:
%%% mode: latex
%%% TeX-master: t
%%% End:
