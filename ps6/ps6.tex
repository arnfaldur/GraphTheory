n\documentclass[a4paper,11pt]{amsart}
\usepackage{amssymb}
\usepackage{graphicx}

\parskip 1ex
\parindent 0 pt

\newcounter{temp}
\newcounter{prob_counter}
\newcounter{sprob_counter}

\newenvironment{problem}
{\begin{list}{{\bf \arabic{prob_counter}}}{
      \usecounter{prob_counter}
      \addtolength{\labelsep}{.6ex}
      \addtolength{\itemsep}{4.3ex}
      \setlength{\leftmargin}{1.4em}}
      \setcounter{prob_counter}{\value{temp}}
}
{\setcounter{temp}{\value{prob_counter}}
  \end{list}
}

\newenvironment{subprob}
{
  \begin{list}{{\bf \alph{sprob_counter}}}{
      \usecounter{sprob_counter}
      \addtolength{\labelsep}{.6ex}
      \addtolength{\itemsep}{.5ex}
      \setlength{\leftmargin}{1.7em}}
}
{\end{list}}

\newenvironment{solution}{\textbf{Solution.}}{\qed}

\newcommand{\rubrik}[1]{\bigskip \begin{center}{\bf #1}\end{center} \medskip}

\newcommand{\NN}{\mathbb{N}}
\newcommand{\ZZ}{\mathbb{Z}}
\newcommand{\QQ}{\mathbb{Q}}
\newcommand{\RR}{\mathbb{R}}




\begin{document}

\pagestyle{empty}
\thispagestyle{empty}

{\small{\sc\noindent
        Name1 ({\tt email1}) and Name2 ({\tt email2})
}}

\rubrik{PROBLEM SET 6 (T-445-GRTH)}

You need to collect $\bf 65$ points to get a full score {\bf but} you cannot get more than {\bf X} points (in total) from a problem section with annotation {\bf max X}.

{\bf Please make sure to:}\\
1. Write your name/email(s) on your work.\\
2. Write your answers in \texttt{{\textbackslash}begin\{solution\} ... {\textbackslash}end\{solution\}} blocks.\\
3. Write clear and concise proofs: points may be deducted for vagueness.




\section{Intersection, Chordal, Outerplanar Graphs ({\bf max 55})}

{\bf Definition.} A planar graph $G$ is \emph{outerplanar} if $G$ has an embedding in the plane in such a way that every vertex is on the boundary of the infinite face.

\begin{problem}
 \item (5 points) Prove that every connected unit interval graph has a Hamiltonian path.
\end{problem}

\begin{solution}
  Let $\widetilde{S} = (I_1,...,I_n)$, be the list of unit intervals that define
  the interval graph $G$, where $I_i = [a_i;b_i] = \{x\in\RR : a_i \leq x \leq
  b_i\}$ ordered so that $a_i \leq a_j$ for $1 \leq i < j \leq n$.

  If $v_i$ is the vertex that corresponds to $I_i$, then the path
  $(v_1,...,v_n)$ is a Hamiltonian path: If $I_i \cap I_{i+1} \neq \emptyset$,
  then $v_i$ and $v_{i+1}$ are connected by an edge, otherwise $G$ is not
  connected.
\end{solution}

\begin{problem}
\item (8 points) Prove that every outerplanar graph on $n$ vertices has at most $2n-3$ edges.
\end{problem}

\begin{solution}
  Let $G$ be n outerplanar graph on $n$ vertices. If $G$ has no inner faces, it
  as a tree, and thus has exactly $n - 1$ edges. Otherwise, we have the
  following:
  \[
    \begin{split}
      3(f-1) + n \leq 2m\\
      3f - 3 + n \leq 2m\\
      f \leq \frac{2m + 3 - n}{3}
    \end{split}
  \]
  Because each interal face has atleast $3$ edges, the external face has at
  least $n$ edges, and every edge belongs to two faces. Plug this into Euler's
  Formula, and we get:
  \[
    \begin{split}
      n - m + f = 2\\
      2 \leq n - m + \frac{2m + 3 - n}{3}\\
      6 \leq 2n - m + 3\\
      m \leq 2n - 3\\
    \end{split}
  \]
\end{solution}

\begin{problem}
\item (10 points) Let $G$ be a graph in which any two simple cycles have at most one vertex in common.
\begin{subprob}
\item Let $B$ be a block in $G$. What can you say about the structure of $B$?
\item Prove that $G$ is an outerplanar graph.
\end{subprob}
\end{problem}

\begin{problem}
\item (7 points) Prove that every unit coin graph on $n > 3$ vertices has at most $3n-7$ edges.
\end{problem}

\begin{solution}
  Let $CG_n$ be the maximally connected unit coin graph with $n$ vertices, $n>3$. By the degree sum formula we know that each vertex $v$ contributes $\frac{d(v)}{2}$ edges to $|E(CG)|$. The maximum degree of any $v \in V(CG)$ is $6$ and thus a maximally connected vertex contributes $3$ to $|E(CG)|$, this means that for any $CG$ we can add a maximally connected vertex and it will not increase the number of edges beyond $3n-7$. In the case of $CG_4$ there would have to be $14$ additional edges for all vertices to be maximally connected. These missing edges are all at the boundary of the graph. It is obvious that the perimiter of $CG_{n-1}$ is smaller or equal in size of $CG_n$. Since we have $|E(CG_4)| = 5 \leq 3*4-7$ the property holds for $CG_4$.
\end{solution}

\begin{problem}
\item (8 points) Prove that for any unit disc graph $G$, $D(G) \le 3(\omega(G)-1)$.
\end{problem}



\begin{problem}
 \item (20 points) A graph $G$ is a {\em split graph} if $V(G)$ can be partitioned into subsets $X,Y$, such that $G[X]$ is a clique and $G[Y]$ is a null graph.
 \begin{subprob}
  \item Prove that split graphs are chordal.
  \item Prove that the complement of a split graph is a split graph.
  %\item Let $G = (X, Y, E)$ be a split graph. Is it true that if $G$ is self-complementary (meaning $G$ is isomorphic to $\overline{G}$), then $|X|=|Y|$?
  \item Find a split graph that is not an interval graph.
 \end{subprob}
\end{problem}

\begin{solution}
  Let $G$ be a split graph, where $X$ is the clique set and $Y$ is the
  independent set. A graph is chordal if it contains no chordless cycles of
  length greater or equal to $3$. Let's assume there exists a chordless
  cycle $C$ on $k$ vertices in $G$. Observe that $C$ may not be composed of
  verteces solely in either $X$, since $G[C]$ would then be a complete graph.

  Let $v$ be one of the vertices such that $v \in C \cap Y$. There must exist
  two vertices $a, b \in N(v) \cap C$, otherwise $C$ wouldn't be a cycle.
  Since $a, b \in X$, and $G[X]$ is complete, there is an edge between $a$
  and $b$. Thus, $G[\{a, b, c\}]$ forms a triangle, and $C$ is not chordless.
  Contradiction!
\end{solution}

\begin{solution}
  Let $G$ be a split graph, where $X$ is the clique set and $Y$ is the
  independent set. Let $\overline{G}$ be the compliment of $G$. Since $G[X]$ is
  a clique, then $\overline{G}[X]$ is a null graph, and since $G[Y]$ is a null graph,
  $\overline{G}[Y]$ is a clique. Ergo $\overline{G}$ is a split graph.
\end{solution}

\begin{solution}
  The split graph composed of $K_4$ and $N_1$ is not an interval graph, because
  it is not chordal; it contains a chordless cycle of length $4$.
\end{solution}


\begin{problem}
 \item (20 points) A graph $G$ is a \emph{threshold graph} if there is a non-negative number $B$ and a non-negative number $a_v$ for each vertex $v\in V(G)$, such that for any subset $U \subseteq V(G)$, $U$ is an independent set \emph{if and only if} $\sum_{v\in U} a_v \le B$.
Use this definition of threshold graphs to prove:
\begin{subprob}
 \item $K_n$ is a threshold graph.
 \item Adding an isolated vertex to a threshold graph gives a threshold graph.
 \item Adding a dominating vertex (a vertex that is connected to
 every other vertex) to a threshold graph gives a threshold graph.
 \item Every threshold graph is a split graph.
\end{subprob}
Note: \emph{Every} threshold graph can be built by repeatedly doing the two operations above (no proof required).
\end{problem}



\section{Powers of Graphs}


\begin{problem}
\item (5 points) Let $\{[a_1, b_1], \ldots, [a_n, b_n]\}$ be an interval representation of a graph $G$.
Show that $\{[a_1, b'_1], \ldots, [a_n, b'_n]\}$, where $$b'_i = \max_j \{b_j : \mbox{$[a_j, b_j]\cap [a_i, b_i] \neq \emptyset$} \},$$ is an interval representation of the graph $G^2$.
\end{problem}



\begin{problem}
\item (10 points) Let $T$ be a tree. Show that $\chi(T^2) = \Delta(T) + 1$.
\end{problem}

\begin{solution}
  Recall the following facts:
  \begin{enumerate}
    \item \textbf{Thm 9.55:} If $T$ is a tree, and $d \in \NN$, then $T^d$ is a
      chordal graph.
    \item \textbf{Cor. 9.37:} Every chordal graph is perfect.
    \item By definition, if a graph $G$ is perfect, then $\chi(G) = \omega(G)$.
  \end{enumerate}

  Let $v$ be a vertex such that $d(v) = \Delta(T)$. Let $\mathit{N} = N(v) \cup \{v\}$ be the
  closed neighbourhood of $v$. Since $v$ is a maximum degree vertex in $T$,
  $T^2[\mathit{N}]$, is a maximum clique in $T^2$. Ergo, we have that
  $\chi(G) = \omega(G) = |N| = \Delta(G) + 1$.
\end{solution}

\section{Perfect Graphs}


\begin{problem}
 \item (8 points) Prove that the complement of an odd cycle $C_{2k+1}$ with $k>1$ is  not a perfect graph.
\end{problem}

\begin{problem}
 \item (10 points) Let $G,H$ be two perfect graphs whose intersection is a complete graph. Prove that $G \cup H$ is perfect.
\end{problem}

\begin{problem}
\item (9 points) Show that perfection is closed neither under edge deletion nor (simple) contractions.
\end{problem}




\section{Claw-Free Graphs}

{\bf Definition.} A \emph{claw} graph is a star with three leaves, i.e. has four vertices, three of them adjacent to the fourth one. A graph is \emph{claw-free} if it doesn't contain a claw as an \emph{induced} subgraph.

{\bf Definition.} The line graph $L(G)$ of a graph $G$ is such that each vertex of $L(G)$ represents an edge of $G$, and two vertices of $L(G)$ are adjacent if and only if their corresponding edges are incident in $G$ (share a vertex). For instance, the line graph of a star is a complete graph.

\begin{problem}
\item (5 points) Prove that the complement of a triangle-free graph is claw-free.
\end{problem}

\begin{solution}
  Let $A_n$ be set of $n$ vertices whose complement is $K_n$. We call $A_n$ an anti-clique of size $n$.
  A claw is a $A_3 \cup v$ where all $u \in A_3$ are connected to $v$. A triangle is $K_3$ and its compliment is $A_3$, thus if there is a claw, there must be a triangle in the complement. Conversely, if there is no triangle, $A_3$ can't be in the complement and thus a claw can't be in the complement.
\end{solution}

\begin{problem}
\item (5 points) Prove that for any graph $G$, the line graph $L(G)$ is claw-free.
\end{problem}

\begin{solution}
  Say there is a claw in $L(A)$. This means that there is an edge $e$ that is adjacent to three edges that are not adjacent to eachother. There must then be two edges $\neq e$, connected to each of the vertices in $e$. There must be a third edge that is connected to a vertex in $e$ which isn't connected to either of the afforementioned edges. This is impossible by the pigeonhole principle.
\end{solution}

\begin{problem}
 \item (7 points) Prove that for any claw-free graph $G$, $\frac{\Delta(G)}{2} \le \chi(G)\le \Delta(G)+1$.
\end{problem}

\begin{problem}
 \item (6 points) Prove that every claw-free interval graph is a unit interval graph (i.e. the class of claw-free interval graphs is precisely the class of unit interval graphs).
\end{problem}






\end{document}


%%% Local Variables:
%%% mode: latex
%%% TeX-master: t
%%% End:
