\documentclass[a4paper,11pt]{amsart}
\usepackage{amssymb}
\usepackage{graphicx}

\parskip 1ex
\parindent 0 pt

\newcounter{temp}
\newcounter{prob_counter}
\newcounter{sprob_counter}

\newenvironment{problem}
{\begin{list}{{\bf \arabic{prob_counter}}}{
      \usecounter{prob_counter}
      \addtolength{\labelsep}{.6ex}
      \addtolength{\itemsep}{4.3ex}
      \setlength{\leftmargin}{1.4em}}
      \setcounter{prob_counter}{\value{temp}}
}
{\setcounter{temp}{\value{prob_counter}}
  \end{list}
}

\newenvironment{subprob}
{
  \begin{list}{{\bf \alph{sprob_counter}}}{
      \usecounter{sprob_counter}
      \addtolength{\labelsep}{.6ex}
      \addtolength{\itemsep}{.5ex}
      \setlength{\leftmargin}{1.7em}}
}
{\end{list}}

\newenvironment{solution}{\textbf{Solution.}}{\qed}

\newcommand{\rubrik}[1]{\bigskip \begin{center}{\bf #1}\end{center} \medskip}

\newcommand{\NN}{\mathbb{N}}
\newcommand{\ZZ}{\mathbb{Z}}
\newcommand{\QQ}{\mathbb{Q}}
\newcommand{\RR}{\mathbb{R}}




\begin{document}

\pagestyle{empty}
\thispagestyle{empty}

{\small{\sc\noindent
        Name1 ({\tt email1@ru.is}) and Name2 ({\tt email2@ru.is})
}}

\rubrik{PROBLEM SET 5 (T-445-GRTH)}

You need to collect $\bf 60$ points to get a full score {\bf but} you cannot get more than {\bf X} points (in total) from a problem section with annotation {\bf max X}.

{\bf Please make sure to:}\\
1. Write your name/email(s) on your work (replace above).\\
2. Write your answers in \texttt{{\textbackslash}begin\{solution\} ... {\textbackslash}end\{solution\}} blocks.\\
3. Write clear and concise proofs: points may be deducted for vagueness.




\section{Independent Sets ({\bf max 30})}

\begin{problem}
 \item (10 points) Find a recursive formula for counting the number of \emph{maximal} independent sets in $P_n$ $(n\ge 1)$. Use it to derive a formula counting the number of maximal independent sets in $C_n$.
\end{problem}

\begin{solution}
  Let $M(G)$ be the set of maximal independent sets of $G$.
  We can represent each element of  $M(P_n)$ as a string of length $n$ where $0 = u \in V(P), u \notin S$ where $S$ is a maximal independent set of $P$ and $1 = v \in V(P), v \in S$. All strings of length $3$ or more, must either start with $010$ or $10$ and from that we can define a reccurence relation. Let $*$ be the concatenation operator. To find the set of maximal independent sets of $P_n$:
  \[M(P_n) = \{10 * e | e \in M(P_{n-2})\} \cup \{010 * e | e \in M(P_{n-3})\}\]
  This gives rise to the recursive formula $pr(n)$:
  \[ pr(n) = \text{number of maximal independent sets in } P_n \]
  \[
    pr(n) =
    \begin{cases}
      1 & \text{if } n = 0\\
      n & \text{if } n \leq 2\\
      pr(n-2) + pr(n-3) & \text{otherwise}
    \end{cases}
  \]
  A similar recurrence relation can be used for $C_n$:
  \[ cr(n) = \text{number of maximal independent sets in } C_n \]
  \[ cr(n) = 1
    \begin{cases}
      1 & \text{if } n = 0\\
      n & \text{if } n \leq 3\\
      2 & \text{if } n = 4\\
      cr(n-2) + cr(n-3) & \text{otherwise}
    \end{cases}
  \]
\end{solution}

\begin{problem}
 \item (8 points) Show that for every simple graph $G$,
\begin{subprob} \item  $\alpha(G)\ge \frac{|V(G)|}{D(G)+1}$.
\item Every \emph{maximal} independent set contains at least $\frac{|V(G)|}{\Delta(G)+1}$ vertices.
\item If $G$ is triangle-free then $\alpha(G)\ge \Delta(G)$.
\end{subprob}
\end{problem}

% Verify
\begin{solution}
  \begin{subprob}
  \item
    Since $\chi(G) \leq D(G) + 1$, we have: $\alpha(G) \geq \frac{|V(G)|}{\chi(G)} \geq \frac{|V(G)|}{D(G)+1}$
  \item
    Every maximal independent set contains at least $\alpha(G)$ vertices by
    definition. Since $\chi(G) \leq \Delta(G) + 1$, we have: $\alpha(G) \geq \frac{|V(G)|}{\chi(G)} \geq \frac{|V(G)|}{\Delta(G)+1}$
  \item
    Let $v$ be a vertex with degree $\Delta(G)$. Observe that $v$'s neighbours
    form an independent set of vertices. Thus, the maximum independent set is of
    size at least $\Delta(G)$.
  \end{subprob}

\end{solution}

\begin{problem}
\item (8 points) Show that:
  \begin{subprob}
  \item  For every simple graph $G$, $\alpha(G)\le |V(G)|-\delta(G)$.
  \item If $G$ has no isolated vertices, then $\alpha(G)\le \frac{|E(G)|}{\delta(G)}$.
  \end{subprob}
\end{problem}
\begin{solution}
  \begin{subprob}
  \item
    By definition the vertex $v \in G, d(v) = \delta(G)$ has no neighbours in the maximum independent set of $G$ so there are at least $\delta(G)$ vertices that are not in the maximum independent set. All other $u \in V(G)$ might be an thus $\alpha(G)\le |V(G)|-\delta(G)$ is true.
  \item
    Each edge $e$ $\forall v, \forall e = \{v,u\}$ where $v$ is in some maximum independent set of $G$ is uniquely connected to $v$ and no other vertex in $G$. All vertices in the maximum independent set have at least $\delta(G)$ unique edges so their number is at least $\alpha(G)\delta(G)$ this number is at most $|E(G)|$ so $\alpha(G)\delta(G) \leq |E(G)|$ which is equivalent to the requested formula.
  \end{subprob}
\end{solution}
\begin{problem}
 \item (15 points) Show that every simple triangle-free graph has an independent set of size at least $\lfloor\sqrt{n}\rfloor$. Find a triangle-free graph that does not have an independent set of size  $\lceil\sqrt{n}\rceil$
\end{problem}

\begin{solution}
  Let G be a simple triangle free graph. Let's observe the degrees of the
  vertices of G. If there exists a vertex $v$ with degree $d(v) \leq \sqrt{n}$,
  then the set of $v$'s neighbours makes up an independent set of atleast
  $\sqrt{n}$ vertices.

  Otherwise, if the maximum degree is $\sqrt{n} -1 \leq \Delta(G)$, we use the
  result from problem 2 section b, to get the following:
  \[
    \sqrt{n} = \frac{n}{(\sqrt{n}-1)+1} \leq \frac{n}{\Delta(G)+1} \leq \alpha(G)
  \]

  An example of a triangle-free graph that does not have an independent set of
  size $\lceil \sqrt{n} \rceil$ (note the ceiling operator, versus the floor
  operator from before), is $C_6$. We can observe that $\alpha(C_6) = 2$, but
  $\lceil \sqrt{6} \rceil = 3$.
\end{solution}

\begin{problem}
\item (5 points) Show that a graph is bipartite if and only if it has a subset of vertices which is both an independent set and a vertex cover.
\end{problem}

\section{Bipartite Matching ({\bf max 30})}



\begin{problem}
 \item (5 points) Let $G=(X\cup Y, E)$ be a simple bipartite graph. Suppose that there is
 a $k \in \NN$ such that $d(x) \ge k \ge d(y)$ for all $x \in X, y \in Y$. Show that $G$
 has a matching saturating $X$ (covering all of $X$).
\end{problem}

\begin{problem}
 \item (12 points) Let $G=(X\cup Y, E)$ be a simple bipartite graph. Let $A$ be the set of vertices in $G$ of
 maximum degree.
 \begin{subprob}
  \item Show that $G$ has a matching saturating $A \cap X$.
  \item Let $W \subseteq X$ and $Z \subseteq Y$.
  Suppose that there is a matching $M$ that saturates $W$, and a matching $N$ that saturates $Z$.
  Prove that there is a matching that saturates $W \cup Z$.
  \item Conclude that there is a matching that saturates $A$.
 \end{subprob}
\end{problem}

\begin{solution}
  \begin{subprob}
  \item
    Let $S = A \cap X$.
    For any $W \subseteq S$,
    there are $\Delta(G)|W|$ edges going into at least $\frac{\Delta(G)|W|}{\Delta(G)} = |W|$ vertices in $Y$.
    Therefore $\forall W, W \subseteq S$ we have $|W| \leq |N(W)|$ which by Hall's Marriage Theorem means that there exists a saturating matching for $S$.
  \item

  \end{subprob}
\end{solution}

\begin{problem}
 \item (6 points) Let $G=(X\cup Y, E)$ be a simple bipartite graph such that $d(x) \ge 1$ for all $x \in X$
 and $d(x) \ge d(y)$ for all $\{x,y\} \in E$ ($x \in X, y \in Y$). Show that $G$ has a matching saturating $X$.
\end{problem}


\begin{problem}
 \item (13 points) Let $G=(X\cup Y, E)$ be a simple bipartite graph containing a perfect matching.
 Prove that there is a vertex $x \in X$, such that for every incident edge $\{x,y\}$, there is a perfect matching
  that contains $\{x,y\}$.
\end{problem}



\begin{problem}
 \item (6 points) Let $X,Y$ be disjoint independent sets in a simple graph $G$, such that $|X|=\alpha(G)$. Prove that $G[X\cup Y]$ has a matching of size $|Y|$.
\end{problem}




\section{More Matching ({\bf max 40})}


\begin{problem}
 \item (6 points) Let $G$ be a simple graph with an even number of vertices such that $d(v) \ge \frac{|V(G)|}{2}$ for every  vertex $v \in V(G)$. Show that $G$ has a perfect matching.
\end{problem}

\begin{problem}
 \item (9 points) Let $T$ be a tree on $n$ vertices with $l$ leaves. Show that $T$ has a matching
 of size at least $\left\lceil \frac{n-l}{2} \right\rceil$.
\end{problem}

\begin{problem}
\item (8 points) For every $k \ge 1$, show that every simple $k$-regular graph has a matching of size at least
$\frac{n}{4 - \frac{2}{k}}$.\\ {\small Hint: show that this bound holds for every maximal matching.}
\end{problem}



\begin{problem}
\item (10 points) Let $M$ be a matching and $X$ be a set of vertices in a graph.
Show that, if $X$ is saturated by $M$, then $X$ is also saturated by some maximum matching.
\end{problem}




\begin{problem}
 \item (10 points) Show that a tree has either one perfect matching or none. For a graph $G$, denote by
 $o(G)$ the number of components of $G$ of odd cardinality. Prove that, for a tree $T$, $T$ has a perfect
 matching, if and only if, $o(T - v) = 1$ for all vertices $v$.
\end{problem}



\begin{problem}
 \item (13 points) Let  $M$ be a maximal matching and $M^*$ be a maximum matching in a graph $G$.
 \begin{subprob}
 \item Show that $|M|\ge |M^*|/2$.
  \item Show that if there is no $M$-augmenting path of length at least three then $|M| \ge 2|M^*|/3$.
 \end{subprob}
\end{problem}





\end{document}


%%% Local Variables:
%%% mode: latex
%%% TeX-master: t
%%% End:
